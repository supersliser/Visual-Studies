\documentclass{article}
\usepackage{graphicx} % Required for inserting images


\usepackage[backend=biber, style=apa, sorting=nyt]{biblatex}
\addbibresource[]{ref.bib}

\title{A study of Melee Weaponry}
\author{Thomas Lower}
\date{January 2024}

\begin{document}

\maketitle

\pagebreak

\tableofcontents

\pagebreak

\section{Introduction}
Action and Adventure is one of the largest Genre's within the Cinematic world in terms of Box Office return \parencite{1}. Furthermore, Adventure can be found in the top 3 of all major gaming platforms \parencite{2}. As such, it makes sense to analyse the core of these genres, weaponry. In this article, I intend to focus upon melee weaponry as I find that these offer a healthy balance between animation potential, efficiency to craft and creative possibility.


In a basic sense, melee weaponry can be divided into 3 key types. These are divided based on the way in which they inflict damage on a target:
\begin{list}{-}{}
    \item Slashing - where the weapon attacks at an angle leaving a large cut.
    \item Bludgeoning - where the weapon uses force to push directly down on an area, delivering concussive damage
    \item Piercing - where the weapon uses force on a small point, piercing vulnerable points on the target
\end{list}
\parencite{bozkir2024just}

Each of these have specific strengths and weaknesses, as well as different symbolisms and meanings attached. Furthermore, weapons could also be categorized in other ways such as by: reach and close ranged, messy and clean or aggressive and defensive. Although, these can often be causative (for example, a weapon is messy because it is aggressive), thus I feel that slashing, bludgeoning and piercing are the best way in which to divide them.

\pagebreak
\section{Parts and Purpose}

\subsection{Offensive Purpose}

\subsection{Defensive Purpose}

\pagebreak

\section{Materials}
%sub sections for different periods?

\pagebreak
\section{Pose and Positioning}



\pagebreak

\section{Weapon Symbolism}
All weapons are associated with a certain concept or meaning behind them, leading to many weapons being used as symbols to display a point or represent a group of people. The most common of which is swords.
\subsection{Sword Symbolism}
As discussed prior, the term sword is in fact a category to describe many different types of weapons \parencite{furat1998brief}, all of which follow the same general shape pattern of: handle → hand guard → single, continuous blade.
All swords would feature a common representation: that of \em{honour}. Within history, most combat would be achieved through armies of \em{levies}, who were donated troops from the \em{barons} of a \em{kingdom}. These troops would not be outfitted by the monarch, but rather, they would outfit themselves with whatever weaponry they could find. Be it scythes, hoes, hatchets or pikes. It would be extremely rare for them to equip a sword as this weapon has no other purpose other than fighting. Thus, it would make no sense for a peasant levy to own a sword. The only people who \em{would} battle with swords would be the monarchs and knights of the army; they would be able to afford a sword as well as afford the time to train with it. This is what causes the sword to be commonly associated with honour, as the monarch was \textit{theoretically} the most honourable soldier in the army. Furthermore, this also allows them to serve as a symbol of hope. While the monarch was alive, it could be assumed that you were still winning - or at the very least \em{not losing} - the battle. Therefore, if you could see a sword, then you could see your monarch

\subsection*{Short sword}
One of the most basic types of sword, would be more accurately described as a \em{short sword} \parencite{mcnab2010swords}. This weapon features a short blade length of no more than 30 cm and often lacks a proper handguard. Thus giving it a sense of aggression as it lacks this key protective implement, requiring the wielder to rely on pure offence in order to overpower their enemies as a posed to being able to fall back to a defensive position when needed. It also requires extreme close range to the target in order to be used effectively, therefore requiring the person holding this weapon to be within direct shot of blood and other gut-like objects when released from the body with this weapon. Which creates an association with it being bloody and brutal as the attacker must be willing to become covered in their enemies blood. Furthermore, it is obvious when somebody wielding this weapon approaches, thus it can be considered bloody and aggressive.

\subsection*{Dagger}
This contrasts greatly to the \em{dagger} which would be considered sly and agile. The symbolism related to such a weapon is often that of deceit and mistrust. This is due to its small size and the fact that this allows it to play a crucial part in assassinations. This is true of the real world, such as the death of Julius Caesar \parencite{caesar} in which a character can be seen holding a dagger. As well as in fictional pieces such as Macbeth in which the character Macbeth famously asks: \begin{quote}
    "Is this a dagger I see before me?"
\end{quote}
before stabbing King Duncan \parencite{macbeth}. Therefore, daggers are often associated with evil as they bring about deceitful death against powerful people.

Daggers also have a second association, that of the occult. Due to the ease of secrecy as well as the precision - both because of the size of the blade, this weapon is often associated with cult-like rituals that require a donation of blood or other sacrifice. The precision of a dagger allows a participant of such a ritual to control the amount of blood used, such that their life is not at risk from blood loss if it is not supposed to be. This association with brutish occult methods is further enhanced because of the need for the wielder to become very close to their target. This, essentially, makes it impossible for a person to avoid getting the blood and innards of the target on themselves, thus requiring a strong constitution or even appreciation of these otherwise disgusting fluids. Thus leading an association with the occult as those who would practice such rituals are often associated to a lack of remorse around the inner parts and guts of others.

\subsection*{Longsword}
However, both of these weapons would be considered to be aggressive, demonstrating the lethality of swords. To search for the representation of honour and protection, one must seek the \em{longsword or broadsword}. This weapon features a much longer blade, such that it could be comfortably held with 2 hands, or wielded with 1 hand if necessary. As well as being featured with a handguard. The size of this weapon allows it to more effectively parry and block, thus allowing it to present as more well-rounded, being capable of slashing at an enemy as well as blocking an attack. This leads to it being the symbol on many coats of arms such as London \parencite{fox1894book} and New York state \parencite{newyorkflag} as it represents adaptability and strength.

\subsection*{Rapier}
Now we can begin to discuss more specialized weapons. To begin, lets start with a personal favourite of mine, the \em{rapier}. This sword is also commonly referred to as a fencing sword and is characterized by an extremely thin blade - such that it will flop and bend under movement of the handle. It also often features a rounded guard or set of rings that covers the hand or fingers \parencite{walker2002rapier12}. This sword acts as a symbol for 1 very specific concept, elegance. The rapier is not a slashing weapon like most swords, but rather, exclusively a piercing weapon \parencite{walker2002rapierNoCut}. Such that it uses straight stabs to cause damage to an individual. These straight stabs are designed to be done extremely quickly in extremely precise spots. To wield this weapon therefore requires both attention and intelligence - to know when and where exactly to stab. These requirements lead to the weapon becoming a dedication. Only somebody who is specifically trained with this weapon will be able to use it properly, anyone else will struggle. With this dedication, a certain grace and elegance follows, in both the biological and mathematical knowledge \parencite{walker2002rapier25} to know where to stab, and the form and positioning that is generated to stab swiftly and effectively. Furthermore, This weapon is exclusively offensive as the flimsy of the blade offers no possible defence. However, its lack of weight offers a different approach. With a lightweight weapon that bends to allow easy aerodynamics, plus forming to allow extremely fast movements for stabs, this weapon lends itself well to the agile. This massively supports the idea of elegance as the agility often results in movements that can be quite shocking and interesting to the eye as a wielder dodges and weaves about heavier and less manoeuvrable types of blade.

The association with elegance also originates from its association with the ruling elite, all of whom would wear this weapon as a part of their civil attire - especially in France and Germany within the Middle Ages \parencite{correa2013history}. The simple fact that all of these people would be commonly carrying this weapon led to a strong association with that of elegance as the elegance of the ruling classes became intrinsically attached to the items they carried.

\subsection*{Greatsword}
Another important weapon to consider is the \em{greatsword or claymore}. This weapon is known exclusively for its massive size. Requiring a minimum of 2 hands to even hope to lift it properly. However, this is not enough in most cases. To lift a claymore would, similar to the rapier, require a large amount of training in order to wield it properly. This weapon perhaps offers the opposite symbolism to the rapier. Where the rapier was a symbol of intelligence and agility, where the wielder only needed to pay attention to their own strikes and dodge the enemies. The claymore relies on sheer brutish strength, and timing strikes and swings with the opponent in order to counteract its complete lack of manoeuvrability (bearing in mind that this weapon is often dragged along the ground as its sheer weight is too much to be carried normally). This ultimately leads to an association with strength, but not the same brutish strength that short swords were known for. No, this is an aged and experienced strength, the strength that comes with training and dedication. A strength mixed with awareness and timing, such that the lack of agility of the blade poses no real issue.

\pagebreak


\printbibliography
\end{document}
