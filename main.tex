\documentclass{article}
\usepackage{graphicx} % Required for inserting images


\usepackage[backend=biber, style=apa, sorting=nyt]{biblatex}
\addbibresource[]{ref.bib}

\title{A study of Melee Weaponry}
\author{Thomas Lower}
\date{January 2024}

\begin{document}

\maketitle

\pagebreak

\tableofcontents

\pagebreak

\section{Introduction}
Action and Adventure is one of the largest Genre's within the Cinematic world in terms of Box Office return \parencite{1}. Furthermore, Adventure can be found in the top 3 of all major gaming platforms \parencite{2}. As such, it makes sense to analyse the core of these genres, weaponry. In this article, I intend to focus upon melee weaponry as I find that these offer a healthy balance between animation potential, efficiency to craft and creative possibility.


In a basic sense, melee weaponry can be divided into 3 key types. These are divided based on the way in which they inflict damage on a target:
\begin{list}{-}{}
    \item Slashing - where the weapon attacks at an angle leaving a large cut.
    \item Bludgeoning - where the weapon uses force to push directly down on an area, delivering concussive damage
    \item Piercing - where the weapon uses force on a small point, piercing vulnerable points on the target
\end{list}
\parencite{bozkir2024just}

Each of these have specific strengths and weaknesses, as well as different symbolisms and meanings attached. Furthermore, weapons could also be categorized in other ways such as by: reach and close ranged, messy and clean or aggressive and defensive. Although, these can often be causative (for example, a weapon is messy because it is aggressive), thus I feel that slashing, bludgeoning and piercing are the best way in which to divide them.

\pagebreak
\section{Parts and Purpose}
\subsection{Swords}
All swords can be broken down into 3 basic components:\begin{list}{-}{-}
    \item A sharp blade which will be used to strike against targets / parry incoming attacks.
    \item An optional cross guard which blocks an incoming strike from hitting the hand of the wielder. Sometimes also referred to as a handguard
    \item A handle with a size relative to the size of the blade - in units of the size of the hand of a human.
\end{list}

The sword of the blade is probably the most crucial - as this is the offensive object of the weapon. The sword blade has multiple different versions with subtle differences.

The sword blade can feature 1 sharpened edge or 2. While the sided blade offers an obvious offensive advantage, this must be balanced with the idea that the blade will therefore require double the maintenance to ensure the blade is properly sharpened, as well as also requiring the blade to be wider. When an object cuts, it uses the sharpened section of the blade to initially separate the object. The width of the blade is then used to push the 2 new parts further apart and ensure a full cut. This then maximizes blood loss and therefore damage to the target. However, this requires the width of the blade to increase at a certain rate as too high an increase will negatively affect the speed of the blade and its ability to cut effectively (try to imagine the different between cutting with a sharp edge, and cutting with the face of a cube). It is because of this that the blade therefore becomes much wider if having multiple sharp edges which had put a strain on materials. It would have also been a heavier weapon if having multiple blades which, in turn, would therefore become more difficult to wield.

Another consideration of the number of sharpened edges is the famous "crossed blades" moment in a filmed fight scene. In this 2 characters wielding longswords will become very close and hold their swords against each other - each pressing the blade against each other in an attempt to simply overpower their opponent. During this situation, a combatant would have a clear advantage if they are able to apply pressure to the back of the blade as Archimedes' research on levering would imply that this is a much more effective place to give force to overpower their opponent \parencite{bunn2017archimedes}. This, however, would only be possible on a single edged sword as otherwise the attacker would almost definitely slice their own hand open if they were to press on the other bladed side of a 2 sided blade.

The point of the blade is also of valid importance as some blades will focus towards a singular spike at the edge, while others will fail to do so. The spike at the end opens the blade up to an additional straight - piercing - strike. This strike is much faster and - due to the lack of angular attack - is much more difficult to parry. However, comes at the cost that the bladed edge close to this point becomes somewhat less effective. Above, I mentioned the width of the blade providing an effective method of properly separating the parts of the target which were cut. If the blade focusses into a point though, this width will have to shrink down towards the end of the blade to allow it to form a point, which renders the bladed edge less effective as it loses this separating ability.

\subsection{Offensive Purpose}

\subsection{Defensive Purpose}

\pagebreak

\section{Materials}
%sub sections for different periods?

\pagebreak
\section{Pose and Positioning}



\pagebreak

\section{Weapon Symbolism}
All weapons are associated with a certain concept or meaning behind them, leading to many weapons being used as symbols to display a point or represent a group of people. The most common of which is swords.
\subsection{Sword Symbolism} \label{swordSymbol}
As discussed prior, the term sword is in fact a category to describe many different types of weapons \parencite{furat1998brief}, all of which follow the same general shape pattern of: handle → hand guard → single, continuous blade.
All swords would feature a common representation: that of \emph{honour}. Within history, most combat would be achieved through armies of \emph{levies}, who were donated troops from the \emph{barons} of a \emph{kingdom}. These troops would not be outfitted by the monarch, but rather, they would outfit themselves with whatever weaponry they could find. Be it scythes, hoes, hatchets or pikes. It would be extremely rare for them to equip a sword as this weapon has no other purpose other than fighting. Thus, it would make no sense for a peasant levy to own a sword. The only people who \emph{would} battle with swords would be the monarchs and knights of the army; they would be able to afford a sword as well as afford the time to train with it. This is what causes the sword to be commonly associated with honour, as the monarch was \textit{theoretically} the most honourable soldier in the army. Furthermore, this also allows them to serve as a symbol of hope. While the monarch was alive, it could be assumed that you were still winning - or at the very least \emph{not losing} - the battle. Therefore, if you could see a sword, then you could see your monarch

\subsection*{Short sword} \label{shortSwordSymbol}
One of the most basic types of sword, would be more accurately described as a \emph{short sword} \parencite{mcnab2010swords}. This weapon features a short blade length of no more than 30 cm and often lacks a proper handguard. Thus giving it a sense of aggression as it lacks this key protective implement, requiring the wielder to rely on pure offence in order to overpower their enemies as a posed to being able to fall back to a defensive position when needed. It also requires extreme close range to the target in order to be used effectively, therefore requiring the person holding this weapon to be within direct shot of blood and other gut-like objects when released from the body with this weapon. Which creates an association with it being bloody and brutal as the attacker must be willing to become covered in their enemies blood. Furthermore, it is obvious when somebody wielding this weapon approaches, thus it can be considered bloody and aggressive.

\subsection*{Dagger}
This contrasts greatly to the \emph{dagger} which would be considered sly and agile. The symbolism related to such a weapon is often that of deceit and mistrust. This is due to its small size and the fact that this allows it to play a crucial part in assassinations. This is true of the real world, such as the death of Julius Caesar \parencite{caesar} in which a character can be seen holding a dagger. As well as in fictional pieces such as Macbeth in which the character Macbeth famously asks: \begin{quote}
    "Is this a dagger I see before me?"
\end{quote}
before stabbing King Duncan \parencite{macbeth}. Therefore, daggers are often associated with evil as they bring about deceitful death against powerful people.

Daggers also have a second association, that of the occult. Due to the ease of secrecy as well as the precision - both because of the size of the blade, this weapon is often associated with cult-like rituals that require a donation of blood or other sacrifice. The precision of a dagger allows a participant of such a ritual to control the amount of blood used, such that their life is not at risk from blood loss if it is not supposed to be. This association with brutish occult methods is further enhanced because of the need for the wielder to become very close to their target. This, essentially, makes it impossible for a person to avoid getting the blood and innards of the target on themselves, thus requiring a strong constitution or even appreciation of these otherwise disgusting fluids. Thus leading an association with the occult as those who would practice such rituals are often associated to a lack of remorse around the inner parts and guts of others.

\subsection*{Longsword}
However, both of these weapons would be considered to be aggressive, demonstrating the lethality of swords. To search for the representation of honour and protection, one must seek the \emph{longsword or broadsword}. This weapon features a much longer blade, such that it could be comfortably held with 2 hands, or wielded with 1 hand if necessary. As well as being featured with a handguard. The size of this weapon allows it to more effectively parry and block, thus allowing it to present as more well-rounded, being capable of slashing at an enemy as well as blocking an attack. This leads to it being the symbol on many coats of arms such as London \parencite{fox1894book} and New York state \parencite{newyorkflag} as it represents adaptability and strength.

\subsection*{Rapier}
Now we can begin to discuss more specialized weapons. To begin, lets start with a personal favourite of mine, the \emph{rapier}. This sword is also commonly referred to as a fencing sword and is characterized by an extremely thin blade - such that it will flop and bend under movement of the handle. It also often features a rounded guard or set of rings that covers the hand or fingers \parencite{walker2002rapier12}. This sword acts as a symbol for 1 very specific concept, elegance. The rapier is not a slashing weapon like most swords, but rather, exclusively a piercing weapon \parencite{walker2002rapierNoCut}. Such that it uses straight stabs to cause damage to an individual. These straight stabs are designed to be done extremely quickly in extremely precise spots. To wield this weapon therefore requires both attention and intelligence - to know when and where exactly to stab. These requirements lead to the weapon becoming a dedication. Only somebody who is specifically trained with this weapon will be able to use it properly, anyone else will struggle. With this dedication, a certain grace and elegance follows, in both the biological and mathematical knowledge \parencite{walker2002rapier25} to know where to stab, and the form and positioning that is generated to stab swiftly and effectively. Furthermore, This weapon is exclusively offensive as the flimsy of the blade offers no possible defence. However, its lack of weight offers a different approach. With a lightweight weapon that bends to allow easy aerodynamics, plus forming to allow extremely fast movements for stabs, this weapon lends itself well to the agile. This massively supports the idea of elegance as the agility often results in movements that can be quite shocking and interesting to the eye as a wielder dodges and weaves about heavier and less manoeuvrable types of blade.

The association with elegance also originates from its association with the ruling elite, all of whom would wear this weapon as a part of their civil attire - especially in France and Germany within the Middle Ages \parencite{correa2013history}. The simple fact that all of these people would be commonly carrying this weapon led to a strong association with that of elegance as the elegance of the ruling classes became intrinsically attached to the items they carried. This also leads to the more practical symbolism of the rapier - which is that of regal influence and upper class status.

\subsection*{Greatsword} \label{greatswordSymbol}
Another important weapon to consider is the \emph{greatsword or claymore}. This weapon is known exclusively for its massive size. Requiring a minimum of 2 hands to even hope to lift it properly. However, this is not enough in most cases. To lift a claymore would, similar to the rapier, require a large amount of training in order to wield it properly. This weapon perhaps offers the opposite symbolism to the rapier. Where the rapier was a symbol of intelligence and agility, where the wielder only needed to pay attention to their own strikes and dodge the enemies. The claymore relies on sheer brutish strength, and timing strikes and swings with the opponent in order to counteract its complete lack of manoeuvrability (bearing in mind that this weapon is often dragged along the ground as its sheer weight is too much to be carried normally). This ultimately leads to an association with strength, but not the same brutish strength that short swords were known for. No, this is an aged and experienced strength, the strength that comes with training and dedication. A strength mixed with awareness and timing, such that the lack of agility of the blade poses no real issue. This practice also leads to a more practical symbolism of the greatsword in the form of age and wisdom - something comparable to the staff of a wizard - due to the practice required by the wielder to achieve strong timing.

It is also worth noting the aggression associated with this weapon, in some part, simply due to its sheer size. But also due to the practice of wielding this weapon. To wield this weapon properly, any movement with the blade should be considered as an attack

\subsection{Spear Symbolism} \label{spearSymbol}
The sword is not the only (neither the most common nor most effective) weapon to cross the hands of fighters. One of the most commonly used throughout history is that of the \emph{spear}. This weapon can have multiple different interpretations based on its specific size. While many other forms of weapons take influence from the spear (mostly appearing to simply take the spear and add extra sections), I will mention most in section \ref{axeSymbol}.

\subsection*{Hunting Spear} \label{huntingSpearSymbol}
The most commonly found spear would likely be the hunting spear, this one is also quite shorter in comparison to its other forms. This spear features very little in the way of artistry of specific design as it is designed to be thrown and expended. Thus, the hunting spear would often be made from a simple wood with a sharp rock or perhaps metal spike attached to the end. It is this design that causes the common association with that of ruggedness or \textit{roughing it}. As the spear is made from natural materials with very little processing or design, it is merely functional and only designed for 1 function at that.

\subsection*{Leaf Spear} \label{leafSpearSymbol}
The leaf spear gathers its namesake from the shape of the blade at the end of the spear, that it is shaped as an oval where 1 becomes concave before forming a point. This weapon is also considerably longer than that of the hunting spear. This weapon would also be more common to be made wholly of metal. Weapon would therefore never be thrown as its size and material would not carry well through the air. Rather, this weapon was used in a skirmish or in place of a sword. The wide blade allowed it to attack in both a piercing and slashing format, while the length of the spear was able to keep an enemy at bay. It was this reason that lead to its adoption by common folk. Because of its increased quality in comparison to that of a sword as well as the ease of production in comparison to that of a blade due to the size of the metal blade being greatly reduced (which could be implied by the lack of "ornament" on common blades \parencite{coffey1893notes}). It is this which then leads us to our goal of ascertaining the symbolism behind this weapon. It can be seen as representation of the common man or worker due to its use by such a people.

\subsection*{Pike}
The pike follows a similar fashion to that of the Leaf Spear. However, rather than a leaf shaped blade, it features more of a straight blade. This would disable it from slashing attacks and limit it purely to that of piercing - as it would not feature a blade on the side of the metal implement with which it could stab \parencite{hayes1943irish}. Despite this, the weapon holds particular affinity with the governmental military world. The weapon is famously the key aspect of Alexander the Great's Macedonian Phalanx \parencite{lazenby2016phalanx}. This featured multiple soldiers, held together tightly by one another, with the first 2 ranks (rows) of soldiers holding their spears in front of the Phalanx. This is considered a legendary tactical manoeuvre. Which helped bring Alexander his namesake "the great". Later, this similar style of grouping was adapted for political representation and used to form that of the marching squad. Thus, the pike holds an extremely strong position with the military as well as the concepts of uniform and unison \parencite{bosworth2010argeads}. It can therefore be stated that the overall symbolism of the pike is a style of unison and uniformity.

\subsection*{Lance}
The lance is a very particular weapon with 2 extremely unique and separate representations. The first is a knightly and regal representation, no doubt a result of the knights who would wield such a weapon into battle on horseback, which would make them extremely visible to all other combatants. This position would often see the knights being decorated with the flags, crests and other such symbolism of the army for which they fought - leading to a strong sense of patriotism to emanate from the knight and that which they held.

The second representation is of pure entertainment. Undoubtably from the sport of jousting. This sport featured 2 people on horse back, running at each other holding wooden lances (it is worth noting that battle-ready lances would have been a metal cone which started at the radius of an adult's torso and ended at a point while the entertainment lance was a wooden cylinder with a smoothly curved tip). The goal was to not fall off your horse when hit by your opponents lance \parencite{fallows2010jousting}. The sport brought crowds who would sit either side and gamble on the victors. This then led to the common association between lances, and entertainment. The relationship is comparable to that of the bat and the sport of baseball, or the ball and the sports of football and basketball.

It is also worth noting that the sport of jousting originates from the knights simply training and practicing their use of the lance for combat - which later evolved into an entertaining sport \parencite{smith2017lance}. 

\subsection*{Glaive}
The glaive is an interesting weapon. The basic design is to stick the full blade of a cleaver (a shortsword with only 1 bladed edge and no pointed tip (see \ref{shortSwordSymbol})) onto the edge of a Hunting Spear (see \ref{huntingSpearSymbol}). This weapon is quite different from the rest of the spear family as it is exclusively a slashing weapon; it has no piercing capability whatsoever. This is interesting because it essentially appears to have a possibly similar representation to that of a leaf spear (see \ref{leafSpearSymbol}), however, with a certain etiquette or skill attached due to the limits of its attack. Furthermore, its fighting style appears to mimic the same general school as that of a greatsword (see \ref{greatswordSymbol}), thus providing a sense of maturity due to the practice of timing strikes right. However, where the greatsword offers simple strength in its manoeuvres, the glaive offers a certain precision due to the much longer handle. This weapon therefore can be said to represent the mindfulness and awareness that the greatsword would represent. However, this is balanced with an increased agility. Thus, allowing this weapon to ultimately present itself, and its wielder as adaptable and strong. It also has a secondary symbolism and that is of balance. When looking upon a glaive, you may notice that the wooden handle takes the same fraction of the weapon as the metallic blade - around \(\frac{1}{2}\) for each component. This therefore allows it to represent the concept of balance and harmony - such that could come with the wisdom of a greatsword.

A conclusion to draw from this is that the Glaive could be understood as the greatsword of the spear family.

\subsection{Axe Symbolism} \label{axeSymbol}
\pagebreak


\printbibliography
\end{document}
