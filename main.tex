\documentclass{article}
\usepackage{graphicx} % Required for inserting images


\usepackage[backend=biber, style=apa, sorting=nyt]{biblatex}
\addbibresource[]{ref.bib}

\title{A study of Melee Weaponry}
\author{Thomas Lower}
\date{January 2024}

\begin{document}

\maketitle

\pagebreak

\tableofcontents

\pagebreak

\section{Introduction}
Action and Adventure is one of the largest Genre's within the Cinematic world in terms of Box Office return \parencite{1}. Furthermore, Adventure can be found in the top 3 of all major gaming platforms \parencite{2}. As such, it makes sense to analyse the core of these genres, weaponry. In this article, I intend to focus upon melee weaponry as I find that these offer a healthy balance between animation portential, efficiency to craft and creative possibility.


In a basic sense, melee weaponry can be divided into 3 key types. These are divided based on the way in which they inflict damage on a target:
\begin{list}{-}{}
    \item Slashing - where the weapon attacks at an anglem leaving a large cut.
    \item Bludgeoning - where the weapon uses force to push directly down on an area, delivering concussive damage
    \item Piercing - where the weapon uses force on a small point, piercing vulnerable points on the target
\end{list}
\parencite{bozkir2024just}

Each of these have specific strengths and weaknesses, as well as different symbolisms and meanings attached. Furthermore, weapons could also be categorized in other ways such as by: reach and close ranged, messy and clean or aggressive and defensive. Although, these can often be causative (for example, a weapon is messy because it is aggressive), thus I feel that slashing, bludgeoning and piercing are the best way in which to divide them.

\pagebreak
\section{Parts and Purpose}

\subsection{Offensive Purpose}

\subsection{Defensive Purpose}

\pagebreak

\section{Materials}
%sub sections for different periods?

\pagebreak
\section{Pose and Positioning}



\pagebreak

\section{Weapon Symbolism}
All weapons are associated with a certain concept or meaning behind them, leading to many weapons being used as symbols to display a point or represent a group of people. The most common of which is swords.
\subsection{Sword Symbolism}
As discussed prior, the term sword is in fact a category to describe many different types of weapons \parencite{furat1998brief}, all of which follow the same general shape pattern of: handle -> hand guard -> single, contineuous blade.
All swords would feature a common representation: that of \textbf{honour}. Within history, most combat would be achieved through armies of \textit{levies}, whom were donated troops from the \textit{barons} of a \textit{kingdom}. These troops would not be outfitted by the monarch, but rather, they would outfit themselves with whatever weaponry they could find. Be it scythes, hoes, hatchets or pikes. It would be extremely rare for them to equip a sword as this weapon has no other purpose other than fighting. Thus, it would make no sense for a peasant levy to own a sword. The only people who \textit{would} battle with swords would be the monarchs and knights of the army; they would be able to afford a sword as well as afford the time to train with it. This is what causes the sword to be commonly associated with honour, as the the monarch was \textit{theoretically} the most honourable soldier in the army. Furthermore, this also allows them to serve as a symbol of hope. While the monarch was alive, it could be assumed that you were still winning - or at the very least \textbf{not losing} - the battle. Therefore, if you could see a sword, then you could see your monarch

\subsection*{Short sword}
One of the most basic types of sword, would be more accurately described as a \textbf{short sword} \parencite{mcnab2010swords}. This weapon features a short blade length of no more than 30cm and often lacks a proper handguard. Thus giving it a sense of agression as it lacks this key protective implement. It also requires extreme close range to the target in order to be used effectively. However, unlike the dagger, it is obvious when somebody wielding this weapon approaches, thus it can be considered bloody and agressive.

\subsection*{Dagger}
This contrasts greatly to the \textbf{dagger} which would be considered sly and agile. The symbolism related to such a weapon is often that of deceit and mistrust. This is due to its small size and the fact that this allows it to play a crucial part in assassinations. This is true of the real world, such as the death of Julius Caesar \parencite{caesar} in which a character can be seen holding a dagger. As well as in fictional pieces such as Macbeth in which the character Macbeth famously asks \begin{quote}
    "Is this a dagger I see before me?"
\end{quote}
before stabbing King Duncan \parencite{macbeth}. Therefore, daggers are often associated with evil as they bring about deceitful death against powerful people.

\subsection*{Longsword}
However, both of these weapons would be considered to be aggressive, demonstrating the lethality of swords. To search for the representation of honour and protection, one must seek the the \textbf{longsword or broadsword}. This weapon features a much longer blade, such that it could be comfortably held with 2 hands, or wielded with 1 hand if necessary. As well as being featured with a handguard. The size of this weapon allows it to more effecitively parry and block, thus allowing it to present as more well rounded, being capable of slashing at an enemy as well as blocking an attack. This leads to it being the symbol on many coats of arms such as London \parencite{fox1894book} and New York state \parencite{newyorkflag} as it represents adaptability and strength.


\pagebreak


\printbibliography
\end{document}
